\documentclass{scrartcl}
\usepackage[latin1]{inputenc}
\usepackage[T1]{fontenc}
\usepackage[ngerman]{babel}
\usepackage{amsmath}
\usepackage{listings}

\title{usbPort - USB Stack f�r Embedded Systeme}
\author{Benedikt Sauter}
\date{29. Oktober 2006}
\begin{document}

\maketitle
\tableofcontents

\section{Einleitung}

\subsection{Anwendungsbereich}
usbPort ist ein USB Stack f"ur Rechnersysteme mit wenig Ressourcen (RAM und CPU, meist Mikrocontroller).
Die Struktur der Software wurde nach der Spezifikation der USB Organisation erstellt.

\subsection{"Ubersicht}
Der USB Stack ist nicht f"ur einen bestimmten USB Host Controller geschrieben, sondern
bietet eine Schnittstelle f"ur USB Host Controller Treiber an. Da der Stack in der Programmiersprache C  geschrieben worden ist,
kann man ihn leicht auf verschiedene Prozessoren portieren.
In der ersten Version gibt es einen Treiber f"ur den Host Controller SL811 von Cypress.
\newline\newline Der Stack kann in verschiedenen Betriebsarten angesteuert werden. F"ur Programme ohne parallele 
Programmpfade, gibt es den sogenannten {\bf Single Modus}. Dieser geht mit dem Arbeitsspeicher sehr sparend um, aber
ist daf"ur bei der bei der Verteilung der Transferrate auf die verschiedenen Verbindungen nicht ausgewogen, wodurch es zu engp"assen kommen
kann. 
Im Gegensatz zum Single Mode gibt es den {\bf Queue Modus}, welcher mehr Speicher ben"otigt, dadurch aber mit einem
flexibleren Algorithmus den Zugriff auf den USB Bus steuert.


\section{USB Stack}

\subsection{Struktur Beschreibung}

Der Stack ist in drei Schichten aufgeteilt. Die Hardware kann man als vierte Schicht ansehen,
wobei die API dieser Schicht durch den eingesetzten USB Baustein vorgegeben ist.
Jede weitere Schicht bietet eine von usbPort angebotene API f"ur die Nachbarschichten an.

\begin{center}
USB Treiber / Direkter Zugriff auf USB Ger"ate \\
USB Kern \\
USB Host Controller Treiber \\
USB Host Controller (Baustein)
\end{center}

\subsection{Verzeichnistruktur}

\begin{itemize}
\item boards 
\item COPYING 
\item core  
\item dbgmonitor 
\item debug.h  
\item devices 
\item doc  
\item drivers  
\item examples 
\item lib  
\item misc 
\item README  
\item types.h 
\item usbspec 
\end{itemize}

\newpage

\subsection{USB Treiber / Direkter Zugriff auf USB Ger"ate}

\subsubsection{USB Treiber API}
Die USB Ger"ate Treiber dienen zur Kommunukation mit den Ger"aten. Es gibt einen allgemeinen
Treiber, mit dem man direkt auf jedes USB Ger"at zugreifen kann. Dazu gibt es eine
Reihe von fertigen Klassen Treibern f"ur verschiedene Ger"ate wie Hubs, Tastaturen, Massenspeicher und einige mehr.
Diese Treiber befinden sich in dem Unterordner drivers/classes.
\newline
\newline
\noindent 
Die API der USB Ger"ate Treiber Schicht:

\begin{figure}[h]
\begin{center}
\begin{tabular}{|l|l|}
\hline
Funktion & Beschreibung\\ \hline
usb\_open() & mitte\\ \hline
usb\_close() & mitte\\ \hline
usb\_set\_configuration() & mitte\\ \hline
usb\_reset() & mitte\\ \hline
usb\_control\_msg() & mitte\\ \hline
usb\_get\_string() & mitte\\ \hline
usb\_get\_string\_simple() & mitte\\ \hline
usb\_get\_descriptor() & mitte\\ \hline
usb\_bulk\_write() & mitte\\ \hline
usb\_bulk\_read() & mitte\\ \hline
usb\_interrupt\_write() & mitte\\ \hline
usb\_interrupt\_read() & mitte\\ \hline
usb\_isochron\_write() & mitte\\ \hline
usb\_isochron\_read() & mitte\\ \hline
\end{tabular}
\end{center}
\caption{USB Treiber API} 
\end{figure}



\newpage

\subsection{USB Kern}

\subsubsection{USB Kern API}

Die prim"are Aufgabe des Kerns ist die Vermittlung von USB Anfragen der Ger"ate Treibern
zu den Host Controller Treibern. Die Anfragen werden mit Hilfe von USB Request Blocks (kurz URB) get"atigt.
\newline
\newline
\noindent
Weitere Aufgaben des Kerns sind:

\begin{itemize}
\item Erkennung und Konfiguartion von neuen Ger"aten
\item Laden und Entladen der Treiber
\item API f"ur Ger"ate Treiber
\item API f"ur Host Controller Treiber
\end{itemize}


\begin{figure}[h]
\begin{center}
\begin{tabular}{|l|l|}
\hline
Funktion & Beschreibung\\ \hline
usb\_core\_init() & mitte\\ \hline
usb\_add\_device() & mitte\\ \hline
usb\_remove\_device() & mitte\\ \hline
usb\_register\_driver() & mitte\\ \hline
usb\_core\_process() & mitte\\ \hline
usb\_alloc\_urb() & mitte\\ \hline
usb\_free\_urb() & mitte\\ \hline
usb\_submit\_urb() & mitte\\ \hline
usb\_fill\_control\_urb() & mitte\\ \hline
usb\_fill\_bulk\_urb() & mitte\\ \hline
usb\_fill\_int\_urb() & mitte\\ \hline
usb\_fill\_iso\_urb() & mitte\\ \hline
\end{tabular}
\end{center}
\caption{USB Kern API} 
\end{figure}

\noindent
Die Hauptfunktion des USB Stacks ist {\bf usb\_core\_process}.
Sie muss zyklisch immer wieder aufgerufen werden, damit der USB Stack
{\it am Leben} gehalten wird. Dies kann "uber verschiedene Weisen geschehen:

\begin{itemize}
\item eine einfache Endlosschleife im Hauptprogramm
\item einem Z"ahlerinterrupt (ausgel"ost von einer RTC)
\item Funktion l"auft in einem Thread (eventuell http://www.sics.se/~adam/pt/)
\end{itemize}

\noindent
Falls der USB Controller "uber eine Interruptleitung verf"ugt,
kann in der dazugeh"origen Routine direkt die Funktion usb\_core\_process
aufgerufen werden. Aber sobald ein Hub angeschlossen wird, reicht der alleinige
Aufruf in der Interruptroutine nicht mehr, es muss ebenfalls daf"ur gesorgt
werden, dass die Funktion regelm"assig aufgerufen wird. Nur so kann
der Kern mitbekommen was auf dem USB Bus hinter dem Hub passiert.

\subsubsection{Datenstrukturen und Arbeitsweise}

Der Kern ist wie er schon vom Namen her verr"at das zentrale Element
in dem USB Stack. Er wird durch die Datenstruktur struct usb\_core dargestellt.

\lstset{language=C}
\begin{lstlisting}
struct usb_core
{
    /* zuletzt verwendete Ger"ate Adresse */
	uint8 nextaddress;
	/* aktuelles Ereigniss des Root Ports am Host */
	uint8 hcd_event;
	/* aktuelle am Bus sich befindende Ger"ate */
	struct usb_device * devices;
	/* angemeldete Treiber */	
	struct usb_drivers * drivers;
};
\end{lstlisting}

\noindent
Genauso wie f"ur den Kern, gibt es f"ur jedes angesteckte Ger"at eine Datenstruktur:

\lstset{language=C}
\begin{lstlisting}
struct usb_device
{
	/* Adresse des Ger"ats */
	uint8 address;
	/* Produkt ID */ 
	uint32 product_id;			
	/* Hersteller ID */
	uint32 vendor_id;		
	/* zugeh"orige Kern */
	struct usb_core *core;		
	/* aktuellen Endpunkte der Konfiguration */
	struct usb_endpoint *endpoints;
};
\end{lstlisting}


\noindent
In der Ger"ate Datenstruktur befindet sich eine Liste mit Endpunkten.
Bei einem Endpunkt wird nur die Fifo Gr"osse gespeichert,
da diese f"ur die Kommunikation wichtig ist.  Es wird in diesem 
Absatz noch darauf eingegangen wo diese genau ben"otigt wird.

\lstset{language=C}
\begin{lstlisting}

struct usb_endpoint
{  
	/* verkettete Liste mit Endpunkten */
    struct usb_endpoint * next;	
	/* Endpunkt Adresse */
	uint8 address;
	/* phsikalische Speicher im Ger"at f"ur diesen Endpunkt */
	uint8 fifosize;
};

\end{lstlisting}


\noindent
Das waren im wesentlichen alle Datenstrukturen, die den Bus
Zustand beschreiben. Zus"atzlich zu diesen gibt es noch
welche die die Nachrichten auf dem USB Bus wiederspiegeln.
\newline
\newline
\noindent
Mit USB Request Blocks (URB) k"onnen Nachrichten versendet werden.
Ein URB sieht wie folgt aus:


\lstset{language=C}
\begin{lstlisting}
struct urb
{
	struct usb_device * dev;
	/* tells if packet is an isochronous */
	u8 iso;
	/* tells if packet is an setup packet */
	u8 setup;
	/* endpoint address */
	u8 endpoint;
	/* buffer for incoming and outgoing data */
	char* buf;
	/* max buffer length */
	u16 buflen;
	/* pointer for function after complete */
	void * complete;
};

\end{lstlisting}

\noindent
Dadurch kann eine Nachricht eindeutig identifiziert werden.
Es ist genau definiert mit welcher "Ubertragungsmethode
die Nachricht an wen in welcher Richtung gesendet wird.
\newline
Die USB Kern API bietet entsprechende Funktionen an,
um URBs zu erstellen und versenden.
\newline
\newline
Im USB Kern werden URBs noch einmal in kleinere Teile
partitioniert, den Transfer Deskriptoren.
USB Host Controller Treiber m"ussen diese Deskriptoren
dann nur noch auf den USB Bus legen und "ubertragen.
\newline
\newline
Tansfer Deskriptoren sehen wie folgt aus:

\lstset{language=C}
\begin{lstlisting}
struct usb_transferdescr
{   
	/* device address */
	u8 address;
	/* endpoint address */
	u8 endpoint;
	/* Interrupt on complete */
	u8 ioc;
	/* PID setup,out or in */
	u8 pid;
	/* if packet is isochronous */
	u8 iso;
	/* Togl Bit select DATA0 or DATA1 */
	u8 togl;
	/* buffer */
	char *buf;
	/* buffer length */
	u16 buflen;
};

\end{lstlisting}

Die Funktion usb\_submit\_urb wandelt URBs
in die entsprechenden Transfer Deskriptoren um.

Daf"ur muss die Nachricht aus dem URB in Endpunkt grosse
St"uecke Zerlegt werden. Beim Aufteilen muss das Togl Bit 
in den einzelnen Transfer Deskriporen gleich
richtig gesetzt werden.

Wenn es sich um ein Setup Paket handelt, muss
das erste Paket speziell mit der PID Setup makiert werden.

Bei einem  UHCI Controller kann man ein spezielles 
Flag Interrupt on Complete setzen. Welches dann daf"ur
sorgt das erst bei einem ensprechenden Paket ein 
Interrupt ausgel"ost wird. Ich will versuchen
dieses Konzept zu "ubernehmen. 

Das heisst man kann bei einem Standard Request z.B.
GetDeviceDescriptor alle 5 Transferdeskriporen absenden.
Der UHCI Controller w"urde jetzt direkt die Puffer
mit den Antwort Bytes f�llen und erst beim letzten TD signalisieren
das er fertig ist.

Der UHCI braucht daf"ur keinen extra Speicher, da er direkt
im Arbeitsspeicher des Computers arbeitet.

Bei Embedded USB Host Controller ist dies leider aber nicht immer 
m"oglich. Der SL811 hat daf"ur zwar 240 Byte Speicher, nur gibt es viele Bausteine die 
"uber so einen internen Speicher nicht verf"ugen.

Das heisst das sammeln der Pakete wird im Host Treiber geschehen.
Dann kann man individuell auf den Baustein eingehen.

Es wird dann zu der Funktion usb\_hcd\_transfer eine benoetigt,
die die Antwort der letzten n TDs abholt. Dazu wird ein zaehler benoetigt
wie oft das Interrupt on Complete Flag nicht gesetzt war.

Wie kommen die Daten jetzt wieder in den Stack zur"uck?

Normalerweise werden im Single Modus die TDs direkt mit den Daten bef"ullt.
Im Queue Modus ohne der oben genannten Idee wuerde man ebenfalls die Daten
direkt in die TDs wieder schreiben.

Die Funktion die die letzten n TDs abholt muss die Daten auf die zugehoerigen TDs verteilen.

Woher weiss man welche TDs zu welcher URB gehoeren?

Eventuell sollte man eine TD Liste im URB speichern.

Gibt es ein System mit den man eindeutige IDs ueber einen besimmten Zeitraum verteilen kann?
Wie kann man merken das eine Grenze uberschritten wird und eine Zahl doppelt hergenommen wird?
Und das ganze sollte mit einem Byte funktionieren.



Warum das Ganze?

Der USB Stack w"urde sonst nahezu 100 Prozent der Prozessorzeit durch warten verbraten.
Auf dem USB Bus wird jede Millisekunde eine SOF generiert. Nach diesem koennen Daten versendet werden.
Das heisst zwischen zwei SOFs kann der USB Baustein selbst"andig arbeiten und man
kann diese Zeit sinnvoll im eigenen Programm nutzen.

\subsubsection{usb\_submit\_urb - im Single oder Queue Modus}


Der Single Modus ist f"ur Umgebungen mit extrem wenig Arbeitsspeicher.
In diesem Modus werden die Nachrichten sofort nacheinander
abgearbeitet. Eigentlich verfolgt ein USB Stack die Strategie,
die verf"ugbare Datenrate fair auf alle Ger"ate am Bus zu verteilen.
So ist z.B. in der USB Spezifikation definiert das f"ur Isochronous- und Interrupt
Transfer bis zu 90 Prozent der Gesamtdatenrate reserviert sind.
Daf"ur braucht man aber entsprechende Warteschlangen um ein faires
aufteilen der Datenrate zu erm"oglichen. Und auf dieses Feature verzichtet
man im Single Modus. Wenn man aber genug Speicherplatz zur Verf"ugung hat
kann man den Queue Modus w"ahlen und hat die eben erw"ahnten Vorteile.
\newline
\newline
\noindent
Im Queue Modus werden die Transferdeskriptoren in USB Kern gesammelt.
Parallel dazu werden immer entsprechend zusammen passende Transferdeskriptoren
f"ur ein Frame gesammelt um so effektiv die Transferrate nutzen zu k"onnen.
\newline
\newline
\noindent
Im Single Modus werden die erzeugen Transferdeskriptoren sofort an
den Host Controller zur "Ubertragung auf dem USB Bus "ubergeben.


\subsubsection{usb\_submit\_urb - Algorithmus}

{\bf Allgemein}
\newline
\begin{enumerate}
\item Pr"ufe ob es sich um ein Setup Request handelt -> if urb->setup==1
\item Falls Setup -> goto Setup Transfer Deskriptor
\item sonst -> pruefe 7. Bit in Endpoint Adresse ob es sich um eine IN oder OUT Pipe handelt
\item Teile Gesamtanzahl der zu "ubertragenden Bytes durch Endpunkt FIFO Gr"oesse
\item Genererie n TDs mit entsprechend gesetztem Togl Bit und if urb->iso makiere TD
\end{enumerate}
\noindent
\newline
{\bf Setup Transfer Deskriptor}

\begin{enumerate}
\item Generiere Setup TD
\item Falls keine Antwort erwartet wird (7. Bit im 8. Byte des Descr) -> PID IN zero data1
\item Sonst Anzahl der zu "ubertragenden Bytes durch Endpunkt FIFO Gr"oesse
\item Genererie n TDs mit entsprechend gesetztem Togl Bit
\item Als Abschluss muss ein OUT zero data1 oder data0 Packet gesendet werden
\end{enumerate}
\noindent
\newline
Nach jedem nicht isochronen Packet muss ein ACK vom Ger"at kommen.
Falls ein NACK kommt, muss das letzte Packet nochmal gesendet werden.


\newpage
\subsection{USB Host Controller Treiber}

\subsubsection{USB Host Controller Treiber API}


Der Host Controller Treiber muss f"ur den USB Kern die folgenden Funktionen anbieten:

\begin{figure}[h]
\begin{center}
\begin{tabular}{|l|l|}
\hline
Funktion & Beschreibung\\ \hline
usb\_hcd\_init() & mitte\\ \hline
usb\_hcd\_event() & mitte\\ \hline
usb\_hcd\_transfer() & mitte\\ \hline
\end{tabular}
\end{center}
\caption{USB Host Controller API}
\end{figure}

\subsubsection{Kommunikation "uber Transfer Deskriptoren}
Die Aufgabe des Host Controllers umfasst die "Ubertragung
von Daten auf dem USB Bus. Die Auftr"age bekommt der Controller
"uber Transfer Deskriptoren. Die Transfer Deskriptoren stellen
Datenstrukturen da, die folgende Informationen beinhalten:

\begin{itemize}
\item Adresse des angesprochenen USB Ger"ats
\item Endpoint
\item ISO (Handelt es sich um ein isochrones Packet?) 
\item PID des Packets (SETUP,IN oder OUT)
\item Toggl Bit
\item Interrupt on complete 
\item Pufferspeicher Adresse
\item Nachrichten L"ange
\end{itemize}

\newpage
\subsubsection{Events vom Host Controller}
Der USB Host Controller Treiber signalisiert dem USB Kern "uber
Events, wenn etwas auf dem Host Controller passiert ist.

\begin{figure}[h]
\begin{center}
\begin{tabular}{|l|l|}
\hline
Ereignisse & Beschreibung\\ \hline
HCD\_EVENT\_NONE & mitte\\ \hline
HCD\_EVENT\_DEVICEFOUND & Neues Ger"at am Host gefunden\\ \hline
HCD\_EVENT\_DEVICEREMOVED & Das Ger"at wurde von Host entfernt\\ \hline
HCD\_EVENT\_NACK & NACK Signal von USB Ger"at empfangen\\ \hline
HCD\_EVENT\_ACK & ACK Signal von USB Ger"at empfangen\\ \hline
HCD\_EVENT\_ISO & Host hat ein isochrones Packet empfangen\\ \hline
HCD\_EVENT\_READY & Controller ist bereit f"ur neue Transaktionen\\ \hline
\end{tabular}
\end{center}
\caption{USB Host Controller Ereignisse} 
\end{figure}

\noindent
Diese Ereignisse sind nur f"ur Ger"ate die direkt am Host Controller
h"angen. Falls mehrer Ger"ate an einem USB Bus betrieben werden sollen,
muss ein USB Hub dazwischen geschaltet werden. Der USB Hub Treiber ist dann
selbst"andig f"ur die Verwaltung der Ger"ate an seinen Ports zust"andig.



\subsubsection{Interrupt Leitung}

Oft haben USB Host Bausteine eine Interrupt Leitung, an der signalisiert
wird wenn sich ein Ereignis im Baustein ereignet hat. In der Interruptroutine
muss das Ereignis am Controller abgefragt werden und danach die Funktion
usb\_core\_process aufgerufen werden.

\lstset{language=C}
\begin{lstlisting}
uint8_t event;

event = usb_hcd_event();
core.hcd_event=event;

/* execute event at once*/
usb_core_process((struct usbcore*)&core);
\end{lstlisting}






\newpage
\section{USB Hub Treiber}



\section{USB Treiber selber schreiben}


\section{Beispiel: Erkennung eines neuen USB Ger"ats}

Das Ger"at wird direkt am USB Host Controller eingesteckt. Nachdem das Ger"at erkannt worden
ist beginnt der USB Kern mit der Enumeration.
\newline
\newline
HCD = Host Controller Drivder
\newline
HC = Host Controller
\newline
CORE = USB Kern
\newline
DRIVERAPI = Treiber API
\newline
TD = Transfer Deskriptor

\begin{enumerate}
\item Ger"at wird angesteckt
\item HC l"ost Interrupt aus
\item Interrupt Routine holt Event und meldet es dem Kern
\item CORE usb\_core\_process startet Enumeration
\item DRIVERAPI GetDeviceDescriptor Request wird erzeugt
\item CORE Request wird abgesendet (usb\_urb\_submit)
\item CORE erzeugt aus dem Requests einzelne Transfer Deskriptoren
\item CORE f"ugt neue TDs in Frame Liste ein
\item HCD erstes TD wird versendet und makiert (SETUP DATA0)	(TD1)

\item HC l"ost Interrupt aus
\item HC meldet das er wieder Betriebsbereit ist
\item HCD zweiter TD wird versendet und makiert (IN DATA1)		(TD2)

\item HC l"ost Interrupt aus
\item HC meldet das er wieder Betriebsbereit ist
\item HCD stoppt Request in dem ein leeres OUT Packet gesendet wird (OUT ZERO DATA0) (TD3)
\item DRIVERAPI SetAddress Request wird erzeugt
\item CORE Request wird abgesendet (usb\_urb\_submit)
\item CORE erzeugt aus dem Requests einzelne Transfer Deskriptoren (TD)
\item CORE f"ugt neue TDs in Frame Liste ein
\item HCD erstes TD wird versendet und makiert (SETUP DATA0)	(TD1)

\item 



\end{enumerate}


\end{document}
